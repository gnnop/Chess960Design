\documentclass[12pt]{article}
\usepackage[margin=0.0in,paperwidth=3in,paperheight=3in]{geometry}
\usepackage{tikz}
\usepackage{xskak}
\usepackage{graphicx}

\pagestyle{empty}

\definecolor{lightgray}{gray}{0.8}
\definecolor{darkgray}{gray}{0.8}

\begin{document}
\centering
\begin{tikzpicture}[x=0.5in,y=0.5in]
  % Set explicit bounding box
  \useasboundingbox (0,0) rectangle (6,6);
  \def\cardsize{6}
  \def\squaresize{\fpeval{6/11}} % Size of each square
  
  \node[align=center] at (3 , 3) {\textbf{CHESS}\\ \textbf{960}};

  % Loop through 4 corners with rotations: 0°, 90°, 180°, 270°
  \foreach \corner in {0,1,2,3} {
    \pgfmathtruncatemacro{\angle}{\corner * 90}
    % Determine corner positions: (0,0), (6,0), (6,6), (0,6)
    \pgfmathsetmacro{\cornerx}{(\corner==1 || \corner==2) ? \cardsize : 0}
    \pgfmathsetmacro{\cornery}{(\corner==2 || \corner==3) ? \cardsize : 0}

    \begin{scope}[rotate around={\angle:(\cornerx,\cornery)}, shift={(\cornerx,\cornery)}]
      % Draw chess setup starting from (0,0) going right
      \foreach \x in {0,...,7} {
        \pgfmathsetmacro{\xpos}{\x * \squaresize + \squaresize/2}
        % Draw squares (2 rows)
        \pgfmathtruncatemacro{\colormod}{mod(\x,2)}
        \ifnum\colormod=0
          \fill[darkgray] (\x*\squaresize, 1*\squaresize) rectangle (\x*\squaresize+\squaresize, 2*\squaresize);
          \fill[lightgray] (\x*\squaresize, 2*\squaresize) rectangle (\x*\squaresize+\squaresize, 3*\squaresize);
        \else
          \fill[lightgray] (\x*\squaresize, 1*\squaresize) rectangle (\x*\squaresize+\squaresize, 2*\squaresize);
          \fill[darkgray] (\x*\squaresize, 2*\squaresize) rectangle (\x*\squaresize+\squaresize, 3*\squaresize);
        \fi
        % Place pieces on squares with white backgrounds
        \node[rotate=\angle] at (\xpos, \fpeval{(6/11)*(3/2)}) {
          \small
          \ifnum\x=0 \WhiteRookOnBlack\fi
          \ifnum\x=1 \WhiteKnightOnWhite\fi
          \ifnum\x=2 \WhiteBishopOnBlack\fi
          \ifnum\x=3 \WhiteQueenOnWhite\fi
          \ifnum\x=4 \WhiteKingOnBlack\fi
          \ifnum\x=5 \WhiteBishopOnWhite\fi
          \ifnum\x=6 \WhiteKnightOnBlack\fi
          \ifnum\x=7 \WhiteRookOnWhite\fi
        };
        \node[rotate=\angle] at (\xpos, \fpeval{(6/11)*(5/2)}) {
          \small
          \ifnum\x=0 \WhitePawnOnWhite\fi
          \ifnum\x=1 \WhitePawnOnBlack\fi
          \ifnum\x=2 \WhitePawnOnWhite\fi
          \ifnum\x=3 \WhitePawnOnBlack\fi
          \ifnum\x=4 \WhitePawnOnWhite\fi
          \ifnum\x=5 \WhitePawnOnBlack\fi
          \ifnum\x=6 \WhitePawnOnWhite\fi
          \ifnum\x=7 \WhitePawnOnBlack\fi
        };
      }
    \end{scope}
  }

\end{tikzpicture}

\end{document}