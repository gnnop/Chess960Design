\documentclass[12pt]{article}
\usepackage[margin=0.0in,paperwidth=3in,paperheight=3in]{geometry}
\usepackage{tikz}
\usepackage{xskak}
\usepackage{graphicx}
\usetikzlibrary{positioning, chains, arrows.meta, shapes.geometric}

\pagestyle{empty}

\definecolor{lightgray}{gray}{0.8}
\definecolor{darkgray}{gray}{0.7}

\begin{document}
\centering
\begin{tikzpicture}[x=0.5in,y=0.5in]
  % Set explicit bounding box
  \useasboundingbox (0,0) rectangle (6,6);
  \def\cardsize{6}
  \pgfmathsetmacro{\squaresize}{6/11} % Size of each square
  
  \node[align=center] at (3 , 3) {\LARGE \textbf{CHESS}\\~\\ \LARGE \textbf{960}};
  
  % The inner list with piece list expanded directly
\def\wholeCard{0/{0/\WhiteRookOnBlack,1/\WhiteKnightOnWhite,2/\WhiteBishopOnBlack,3/\WhiteQueenOnWhite,4/\WhiteKingOnBlack,5/\WhiteBishopOnWhite,6/\WhiteKnightOnBlack,7/\WhiteRookOnWhite}/0/7/4,1/{0/\WhiteRookOnBlack,1/\WhiteKnightOnWhite,2/\WhiteBishopOnBlack,3/\WhiteQueenOnWhite,4/\WhiteKingOnBlack,5/\WhiteBishopOnWhite,6/\WhiteKnightOnBlack,7/\WhiteRookOnWhite}/0/7/4,2/{0/\WhiteRookOnBlack,1/\WhiteKnightOnWhite,2/\WhiteBishopOnBlack,3/\WhiteQueenOnWhite,4/\WhiteKingOnBlack,5/\WhiteBishopOnWhite,6/\WhiteKnightOnBlack,7/\WhiteRookOnWhite}/0/7/4,3/{0/\WhiteRookOnBlack,1/\WhiteKnightOnWhite,2/\WhiteBishopOnBlack,3/\WhiteQueenOnWhite,4/\WhiteKingOnBlack,5/\WhiteBishopOnWhite,6/\WhiteKnightOnBlack,7/\WhiteRookOnWhite}/0/7/4}
  
  % Loop through 4 corners with rotations
  \foreach \corner/\pieceSetup/\lRook/\rRook/\king in \wholeCard {
    \pgfmathtruncatemacro{\angle}{\corner * 90}
    % Determine corner positions
    \pgfmathsetmacro{\cornerx}{(\corner==1 || \corner==2) ? \cardsize : 0}
    \pgfmathsetmacro{\cornery}{(\corner==2 || \corner==3) ? \cardsize : 0}

    \begin{scope}[rotate around={\angle:(\cornerx,\cornery)}, shift={(\cornerx,\cornery)}]
      % Draw chess setup
      \foreach \x/\piece in \pieceSetup {
        \pgfmathsetmacro{\xpos}{\x * \squaresize + \squaresize/2}
        % Draw squares
        \pgfmathtruncatemacro{\colormod}{mod(\x,2)}
        \ifnum\colormod=0
          \fill[darkgray] (\x*\squaresize, 1*\squaresize) rectangle (\x*\squaresize+\squaresize, 2*\squaresize);
          \fill[lightgray] (\x*\squaresize, 2*\squaresize) rectangle (\x*\squaresize+\squaresize, 3*\squaresize);
        \else
          \fill[lightgray] (\x*\squaresize, 1*\squaresize) rectangle (\x*\squaresize+\squaresize, 2*\squaresize);
          \fill[darkgray] (\x*\squaresize, 2*\squaresize) rectangle (\x*\squaresize+\squaresize, 3*\squaresize);
        \fi
        % Place pieces
        \node[rotate=\angle] at (\xpos, {(6/11)*(3/2)}) {\small \piece};
        \node[rotate=\angle] at (\xpos, {(6/11)*(5/2)}) {
          \small
          \ifnum\colormod=0 \WhitePawnOnWhite\else\WhitePawnOnBlack\fi
        };
      }
     
      % Castling indicators
      \foreach \startfile/\endfile/\offset/\direction in {\king/2/0.3/->,\king/6/0.5/->,\lRook/3/0.4/->,\rRook/5/0.6/->} {
        \pgfmathsetmacro{\startx}{\startfile * \squaresize + \squaresize/2}
        \pgfmathsetmacro{\neary}{1*\squaresize}
        \pgfmathsetmacro{\endx}{\endfile * \squaresize + \squaresize/2}
        \pgfmathsetmacro{\midy}{\offset*\squaresize}
    
        \draw[\direction,line width=0.3mm,rounded corners=5pt] (\startx, \neary) -- (\startx, \midy) -- (\endx, \midy) -- (\endx, \neary);
      }
    \end{scope}
  }

\end{tikzpicture}

\end{document}